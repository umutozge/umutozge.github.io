\documentclass[sans,10pt,a4paper]{moderncv}

%\newcommand{\itit}{\item\hspace{10pt}$\bullet$ }
%\renewcommand{\listitemsymbol}{$\bullet$} % change the symbol for lists

% moderncv themes
%\moderncvtheme[blue]{casual}                 % optional argument are 'blue' (default), 'orange', 'red', 'green', 'grey' and 'roman' (for roman fonts, instead of sans serif fonts)
\moderncvtheme[blue]{classic}                % idem

\usepackage{eurosym}
\usepackage[utf8]{inputenc}                   % replace by the encoding you are using
\usepackage{verbatim}
\usepackage[scale=0.85]{geometry}
%setlength{\hintscolumnwidth}{3cm}						% if you want to change the width of the column with the dates
%\AtBeginDocument{\setlength{\maketitlenamewidth}{6cm}}  % only for the classic theme, if you want to change the width of your name placeholder (to leave more space for your address details
\AtBeginDocument{\recomputelengths}                     % required when changes are made to page layout lengths

% personal data
\firstname{Umut}
\familyname{\"{O}zge}
%\title{Resum� title (optional)}               % optional, remove the line if not wanted
%\address{Harvard University, William James Hall, Room 1074}{33 Kirkland Street, 02138 Cambridge, MA, USA }
\address{Informatics Institute}{Middle East Technical University}{Dumlupinar
Bulv.\ 1, 06800, Ankara Turkey}

\phone{+90-530-142-6850} 
%\phone{+1-617-642-8342} 
%\mobile{+49 (0)162 861-4004}                    
%\fax{fax (optional)}
\email{tumuum@gmail.com}
\homepage{https://umutozge.github.io/}

%\extrainfo{Permanent Address: Kipta\c{s} Bloklar{\i} 1/22, 34450 Istanbul TURKEY} % optional, remove the line if not wanted
%\photo[64pt]{picture}                         % '64pt' is the height the picture must be resized to and 'picture' is the name of the picture file; optional, remove the line if not wanted
%\quote{Some quote (optional)}                 % optional, remove the line if not wanted

%\nopagenumbers{}                             % uncomment to suppress automatic page numbering for CVs longer than one page


%----------------------------------------------------------------------------------
%            content
%----------------------------------------------------------------------------------
\begin{document}
\maketitle


\section{Personal Information}
\cvcomputer{Date of Birth}{26 July 1976}{Place of Birth}{Ankara, Turkey}
\cvcomputer{Citizenship}{Turkish}{Marital Status}{Married}

\section{Education}
\cventry{Sep~2010}{PhD}
{Cognitive Science, Middle East Technical University}
{}
{Ankara, Turkey. \newline Dissertation Title: ``Grammar and Information: A Study of Turkish
Indefinites''
\newline Supervisor: Prof. Dr. Cem Boz\c{s}ahin}{}

\cventry{Jan~2003}{MSc}
{Cognitive Science, Informatics Institute, Middle East Technical University}
{}
{Ankara, Turkey.
\newline Dissertation Title: ``A Tune-Based Account of Turkish Information
Structure''
\newline Supervisor: Prof. Dr. Cem Boz\c{s}ahin}{}

\cventry{Jan~1999}{BSc}
{Chemical Engineering, Middle East Technical University}
{}
{Ankara, Turkey.
\newline Graduation Project Title: ``Computer Aided Design of a
Nitric Acid Distillation Column''}  
{}

\section{Academic and Professional Experience}

\cventry{Jan~2016-}
{Assistant Professor}
{Cognitive Science Department, Informatics Institute, Middle East Technical University}
{Ankara, Turkey}{}{}
\cventry{Sep~2015-Dec~2015}
{Instructor}
{Cognitive Science Department, Informatics Institute, Middle East Technical University}
{Ankara, Turkey}{}{}
\cventry{Mar~2015-Aug~2015}
{Post-doctoral visitor}
{Cognitive Science Department, Informatics Institute, Middle East Technical University}
{Ankara, Turkey}{}{}

\cventry{Apr~2014--Feb~2015}
{Post-doctoral researcher}
{Institute of German Language and
Literature 1, University of Cologne}
	{Cologne, Germany}{}
{DFG Project:``Indefinites in Discourse'', \\ PI: Prof.\ Dr.\ Klaus von
Heusinger}

\cventry{Jan~2013--Feb~2014}
{Post-doctoral visitor}
{Harvard University}
{Cambridge, USA}{}
{Host: Prof.\ Dr.\ Jesse Snedeker}

\cventry{Jun~2011--Dec~2012}
{Post-doctoral researcher}
{SFB 732:``Incremental Specification in Context'', Institute of General Linguistics, University of Stuttgart}
{Stuttgart, Germany}{}
{Project C2: ``Case and Referential Context''\\ PI: Prof.\ Dr.\ Klaus von Heusinger}


\cventry{Jan~2011--May~2011}
{Post-doctoral fellow}
{Department of Psychology, Harvard University}
{Cambridge, USA}{}
{Host: Prof. Dr. Jesse Snedeker}

\cventry{Sep~2010-Jan~2010}
{Instructor}
{Informatics Institute, Middle East Technical University}
{Ankara, Turkey}{}
{
COGS 515 Artificial Intelligence for Cognitive
Science \\
COGS 501 Linguistics and Formal Languages (co-taught)
}


\cventry{Oct~2007--Oct~2008}
{Visiting Post-graduate Student}
{Institute for Communicating and Collaborative Systems, University of
\mbox{Edinburgh}}
{Scotland, UK}{}
{Host: Prof. Dr. Mark Steedman}

\cventry{Sep~2000--Sep~2007}
{Research \& Teaching Assistant}
{Informatics Institute, Middle East Technical University}
{Ankara, Turkey}{}{
METU Turkish Corpus Project (\url{www.ii.metu.edu.tr/-corpus})\\
METU-Sabanc\i Turkish Dependency Treebank Project
(\url{www.ii.metu.edu.tr/-corpus/treebank/})\\
Developed a corpus query workbench in Java, used by many Turkish linguists.\\
TA for courses \emph{Theoretical Linguistics}, \emph{Linguistics and Formal
Languages}, \emph{Programming and Logic}, \emph{Computational Linguistics}\\
Co-taught \emph{Programming and Logic}
% 	\vspace{1pt}
% \textit{METU Turkish Corpus Project (www.ii.metu.edu.tr/-corpus)}
% 	\begin{itemize}
% 			\itit Supervision and coordination of the XCES annotators. 
% 			\itit  Development of text processing scripts to enhance the building of the
% 			corpus. 
% 			\itit Development of a corpus query workbench for METU Turkish Corpus in Java. 
% 	\end{itemize}
% 	\vspace{1pt}
% \textit{METU-Sabanc\i Turkish Dependency Treebank Project (www.ii.metu.edu.tr/-corpus/treebank/)}
% 	\begin{itemize} 
% 			\itit Worked with other project members in developing the dependency scheme for Turkish. 
%  			\itit Dependency annotation. 
% 	\end{itemize}
% 	\vspace{1pt}
% \textit{Assisted Assist. Prof. Dr. Bilge Say (see References) in:}
%  	\begin{itemize}
% 			\itit COGS 501 Linguistics and Formal Languages 
% 			\itit COGS 502 Programming and Logic 
% 	\end{itemize}
% 	\vspace{1pt}
% \textit{Assisted Assoc. Prof. Dr. Cem Boz\c{s}ahin (see References)
% in:}
% 	\begin{itemize}
%   			\itit CENG 563 Computational Linguistics
% 			\itit COGS 532 Theoretical Linguistics
% 	\end{itemize}
}

\cventry{Feb~1999--Feb~2000}
{Chemical Engineer}
{G\"ural Cam Art \& Craft Glass Factory}
{K\"utahya, Turkey}{}{
Combustion engineer, enterprise resource planning (BAAN).
}
%\cventry{year--year}{Job title}{Employer}{City}{}{Description}

\section{Research Interests}
\cvline{}{Logical/formal and empirical (corpus, experimental) approaches to natural language
semantics and pragmatics (focusing on the semantics and pragmatics of nominal
expressions, information structure, and its relation to intonation).} 
\cvline{}{Computation of compositional semantics in the form of Discourse
Representation Structures via Combinatory Categorial Grammar.} 
\cvline{}{Combinatory Logic as a model of computation in cognition and language.}
\cvline{}{Foundations of natural language semantics (the tension and interaction
between the principle of contextuality and the principle of compositionality).} 
	
\newcommand{\SortName}[1]{}
\begin{thebibliography}{}

% \bibitem{lan}
% \"Ozge, U. (in prep.).
% \newblock On the ``strength'' of indefinites.
% \newblock For submission to {\em Language}. 
% 
% \bibitem{lan}
% \"Ozge, U. and von Heusinger, K. (in prep.).
% \newblock Varieties of inferrable indefinites.
% \newblock For submission to {\em Journal of Semantics}. 


\bibitem{2016}
\"Ozge, U., \"Ozge, D., and von Heusinger K. (2016).
\newblock Strong indefinites in Turkish: Salience structure and referential
persistence. In A. Holler and Katja Suckow (eds.) {\em Empirical Perspectives on Anaphora
Resolution.} Linguistische Arbeiten, de Gruyter.


\bibitem{ozge14}
\"Ozge, U. (2014).
\newblock Review of \textit{{C}ombinatory {L}inguistics} by {C}em
  {B}oz\c{s}ahin.
\newblock {\em Dilbilim Ara\c st\i rmalar\i\ Dergisi\/}, {2014/1},
  103--111.

\bibitem{2013b}
\"Ozge, U. (ed.) (2013a).
\newblock {\em Proceedings of the 8th Workshop on Altaic
Formal Linguistics.} MIT Working Papers in Linguistics, Cambridge, MA.

\vspace{5pt}

\bibitem{2012b}
\"Ozge, U. (2013b).
\newblock What does it mean for an indefinite to be presuppositional? 
\newblock In G.\ Bezhanishvili, S.\ L\"obner, V.\
Marra, F.\ Richter (eds.) \textit{Logic, Language, and Computation - 9th
International Tbilisi Symposium on Logic, Language, and Computation,
TbiLLC 2011}, pages 138--154. Revised Selected
Papers. Lecture Notes in Computer Science 7758, Springer, Berlin.

\vspace{5pt}

\bibitem{2012b}
\"Ozge, U. (2013c).
\newblock "Strong" Indefiniteness and Topicality. 
\newblock In E. Chemla, V. Homer, G. Winterstein  (eds.) Proceedings of Sinn und Bedeutung 17,
pages 399-408, published online at semanticsarchive.net

\vspace{5pt}


\bibitem{2012a}

\"Ozge, U. (2012).
\newblock Notes on focus projection in Turkish. 
\newblock In \'E. Kincses-Nagy and M\'onika Biacsi  (eds.) {\em The Szeged
Conference: The proceedings of the 15th International Conference on Turkish
Linguistics}, pages 389--399, University of Szeged, Szeged, Hungary. 


\bibitem{ozge11}
\"Ozge, U. (2011).
\newblock {T}urkish indefinites and accusative marking.
\newblock In A.~Simpson (ed.) {\em Proceedings of the 7th Workshop on Altaic
  Formal Linguistics, MITWPL \#63\/}, pages  253--267. MITWPL, Cambridge, MA.

\vspace{5pt}

\bibitem{ozge10}
\"Ozge, U. (2010).
\newblock {\em Grammar and Information: A Study of Turkish Indefinites.}
\newblock Ph.D. thesis, Middle East Technical University.
\newblock URL: \url{etd.lib.metu.edu.tr/upload/12612641/index.pdf}

\vspace{5pt}

\bibitem{ozgebozsahin10}
\"Ozge, U. and Bozsahin, C. (2010).
\newblock Intonation in the grammar of {T}urkish.
\newblock {\em Lingua\/}, {120}, 132--175.

\vspace{5pt}

\bibitem{ozge09}
\"{O}zge, U. (2009).
\newblock Linear order, focus, and pronominal binding in {T}urkish.
\newblock In S.~Ay, O.~Ayd{\i}n, I.~Ergen\c{c}, S.~G\"{o}kmen,
  S.~\.{I}\c{s}sever, and D.~Pe\c{c}enek (eds.) {\em Essays on {T}urkish
  {L}inguistics: Proceedings of the 14th International Conference on {T}urkish
  Linguistics,} pages  131--141, Harrasowitz
  Verlag, Wiesbaden. 

\vspace{5pt}

\bibitem{ozge06}
\"{O}zge, U. (2006).
\newblock Word order, prosody and information structure in {T}urkish.
\newblock In S.~Ya\u{g}c{\i}o\u{g}lu and A.~C. De\u{g}er (eds.) {\em
  Advances in {T}urkish {L}inguistics: Proceedings of the 12th International
  Conference on {T}urkish Linguistics\/}, pages  25--35. Dokuz Eyl\"{u}l
  University Press, \.{I}zmir.

\vspace{5pt}

\bibitem{ozgesay04}
\"{O}zge, U. and Say, B. (2004).
\newblock Development of a corpus workbench for the {METU} {T}urkish {C}orpus.
\newblock In M.~T. Lino, M.~F. Xavier, F.~Ferreira, R.~Costa, and R.~Silva,
  (eds.) {\em Proceedings of the 4th Language Resources and Evaluation
  Conference, Lizbon, Portugal\/}, pages  223--225.

\vspace{5pt}

\bibitem{sayetal04}
Say, B., Zeyrek, D., Oflazer, K., and \"{O}zge, U. (2004).
\newblock Development of a corpus and a treebank for present-day written
  {T}urkish.
\newblock In K.~\.{I}mer and G.~Do\u{g}an (eds.) {\em Current Research in
  {T}urkish {L}inguistics: Proceedings of the 11th International Conference on
  {T}urkish Linguistics\/}, pages  183--192. Eastern Mediterranean University
  Press, Gazima\u{g}usa.

\end{thebibliography}
% \nocite{ozgeetal13,ozge12a,ozge12b,ozgeetal12,ozge11,ozgebozsahin10,ozge10a,ozge09,ozge06,ozgesay04,sayetal04}
% \bibliographystyle{natgig}
% \bibliography{ozge}


\renewcommand{\refname}{Recent Conference \& Workshop Presentations}
\begin{thebibliography}{}

\bibitem{}
\"Ozge, U. and von Heusinger, K. (2017).
\newblock Inferrable and partitive indefinites in topic position 
\newblock Paper presented at DGfS 2017, 8-10 March 2017, Saarland University.

 
\bibitem{}
Torregrossa, J., \"Ozge, U. (2016).
\newblock Varieties of D-linking: A view from Italian and Turkish.
\newblock Poster in Going Romance 48, 8-10 December 2016, University of
Frankfurt .


\bibitem{}
\"Ozge, U. and von Heusinger, K. (2014).
\newblock Two types of inferrable indefinites. 
\newblock Poster in RefNet Workshop on Psychological and Computational Models of
Reference Comprehension and Production, Edinburgh, August 2014.

\vspace{5pt}

\bibitem{}
\"Ozge, U. (2014).
\newblock Turkish {DOM} and presuppositionality.
\newblock Mini-workshop on ``Case-marking and word order in Turkic'', Cologne,
May 23, 2013.

\vspace{5pt}

\bibitem{ozgeetal13}
\"Ozge, D., \"Ozge, U., and von Heusinger, K. (2013).
\newblock Turkish optional case marking as an indicator of discourse salience.
\newblock DGfS Workshop ``Information structural evidence in the race for salience'', G\"ottingen,
  March 2013.

\vspace{5pt}

\bibitem
  {ozgeetal12}
\"Ozge, D., \"Ozge, U., and von Heusinger, K. (2012).
\newblock Discourse Structuring Potential of optional accusative marking in
  {T}urkish.
\newblock Poster in Architectures and Mechanisms for Language Processing (AMLAP). Italy, Sep 2012.

\vspace{5pt}

\bibitem{ozge12b}
\"Ozge, U. (2012).
\newblock On the ``strength'' of indefinites.
\newblock Sinn und Bedeutung 17, Paris, Sep 2012.

\vspace{5pt}


\bibitem{dummy2}
von Heusinger, K. and \"Ozge, U. (2011). 
\newblock Specificity, referentiality and discourse prominence. 
\newblock International Workshop ``Dimensions of Grammar'' in honor of Paul Kiparsky, University of Konstanz, 2 August 2011. 

\vspace{5pt}

\bibitem{dummy3}
\"Ozge, U. (2011). 
\newblock Notes on distributivity in Turkish plural indefinites. 
\newblock Workshop  ``\,`Quantification' by Anna Szabolcsi'', University of Stuttgart, 15 July 2011.

\vspace{5pt}

\bibitem{dummy4}
\"Ozge, U. (2010). 
\newblock On Turkish singular indefinites. 
\newblock Workshop ``Specificity from Theoretical and Empirical Perspectives'', University of Stuttgart, 31 August 2010.

\end{thebibliography}

\renewcommand{\refname}{Recent Invited Talks}
\begin{thebibliography}{}

\bibitem{dummy1}

On the semantics of Turkish accusative indefinites.\\
University of Cologne, Cologne, 4 Feb 2014.

On the ``strength'' of indefinites: Topicality, scope, presuppositionality.\\
Language \& Cognition Seminar, Harvard University, Cambridge, 10 Dec 2013. 

On the ``strength'' of indefinites.\\
Ankara Linguistic Circle, Middle East Technical University, Ankara, 28 Dec 2012.

(with D. \"Ozge) Accusative indefinites in Turkish: Forward and backward
perspectives. \\
Bo\u{g}azi\c ci University, Istanbul, 20 Dec 2012. 


On the ``strength'' of indefinites: A view from Turkish.\\ 
Heinrich-Heine University, D\"usseldorf, 9 Feb 2012. 

CCG and flexible word order: An introduction\\
University of G\"ottingen, G\"ottingen, 17 Jan 2012.

\end{thebibliography}


\section{Grants}
\cvline{Apr~2014-Apr~2017}{Research Project Grant awarded by German Science
Foundation (DFG)  for the  project
``Indefinites in Discourse'' (\euro 400K/3yrs, GZ: HE 6893/14-1), co-written
with Klaus von Heusinger (PI).}
\cvline{Sep~2007--Sep~2008}{\small Doctoral Research Grant
awarded by The Scientific and Technological Research Council of Turkey (TUBITAK).}
%\cvline{August 2002}{European Summer School on Logic, Language
%and Information Student Scholarship.}

% \section{Courses and Certificates}
% \cvline{12th~Jun--23rd~Jun~2006}{LOT Summer School in Linguistics,
% Amsterdam, Netherlands.}
% \cvline{13th~August--24th~August~2002}{European Summer School on Logic, Language
% and Information, Trento, Italy.}


% \section{Memberships}
% \cvline{2006--Present}{\small Ankara Linguistic Circle}


\section{Organizational Duties}

\cvline{May~23, 2014}{Organization of a mini-workshop on ``Case marking and word
order in Turkic'', University of Cologne}

\cvline{Jun~2012--}{Design and co-supervision of a publication database on
semantics and related topics, University of Stuttgart, University of Cologne
(with Chiara Gianollo).}

\cvline{May~18--20th, 2012}{Co-organization of the 8th Workshop on Altaic Formal
Linguistics, University of Stuttgart (with Klaus von Heusinger and Jaklin
Kornfilt).} 

\cvline{Sep~2009--Jan~2010}{Co-organization of Ankara
Linguistic Circle Seminars at the Middle East Technical University
Cognitive Science Department (with Duygu \"{O}zge).}


\section{Languages}
\cvlanguage{Turkish}{Native}{}
\cvlanguage{English}{Fluent}{}
\cvlanguage{German}{Intermediate (A2/B1)}{}


\section{Computer skills}
\cvline{Languages}{Advanced: Java, Python, Prolog; Intermediate: shell scripting, Haskell, Scheme, R}
\cvline{Software}{Vim (incl.\ scripting), \LaTeX, Praat, MMAX2, E-prime}
%\cvline{Operating Systems}{GNU/Linux, MAC OSX}

% \section{References}
% 
% \cvline{}{The following people are familiar with my research in semantics:}
% 
% \cventry{}{Klaus von Heusinger}
% {University of Cologne}
% {}
% {}
% {\url{klaus.vonheusinger@uni-koeln.de}}
% 
% \cventry{}{Hans Kamp}
% {University of Stuttgart}
% {}
% {}
% {\url{hans.kamp@ims.uni-stuttgart.de}}
% 
% \cventry{}{Anastasia Giannakidou}
% {University of Chicago}
% {}
% {}
% {\url{ginnaki@uchicago.edu}}
% 
% \vspace{20pt}
% 
% \cvline{}{The following people are familiar with my organizational skills:}
% 
% \cventry{}{Klaus von Heusinger}
% {University of Cologne}
% {}
% {}
% {\url{klaus.vonheusinger@uni-koeln.de}}
% 
% \cventry{}{Sabine Iatridou}
% {MIT}
% {}
% {}
% {\url{iatridou@mit.edu}}
% 
% \cventry{}{Jaklin Kornfilt}
% {Syracuse University}
% {}
% {}
% {\url{kornfilt@syr.edu}}
% 
% \cventry{}{Shigeru Miyagawa}
% {MIT}
% {}
% {}
% {\url{miyagawa@mit.edu}}
\vspace{10pt}
{\em Last updated: May, 2017}

% \section{References}
% 
% \cventry{}{Prof. Dr. Klaus von Heusinger}
% {Institute of German Language and Literature I}
% {}
% {University of Cologne, Albertus Magnus Platz D-50923 Cologne}
% {e-mail: Klaus.vonHeusinger@uni-koeln.de tel: +49-(0)221-470-4884}
% 
% \cventry{}{Prof. Dr. Mark Steedman}
% {Institute for Language, Cognition and Computation}
% {}
% {University of Edinburgh, Edinburgh, UK}
% {e-mail: steedman@inf.ed.ac.uk, tel: +44-131-6504631}
% 
% \cventry{}{Prof. Dr. Jaklin Kornfilt}
% {Languages, Literatures, and Linguistics}
% {}
% {Syracuse University 305 HB Crouse Hall}
% {e-mail: kornfilt@syr.edu, tel: +1-315-443-5375}
% 
% \cventry{}{Prof. Dr. Cem Boz\c{s}ahin}
% {Department of Cognitive Science}{} 
% {Middle East Technical University 06531 Ankara,Turkey}
% {e-mail: bozsahin@metu.edu.tr, tel: +90-312-210-5580}
% 
\end{document}
